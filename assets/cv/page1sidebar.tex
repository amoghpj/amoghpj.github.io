\cvsection{Skills}
\begin{itemize}
\item \textbf{Languages} Python, MATLAB,  Bash, C++, Emacs-lisp
\item \textbf{Laboratory} Tissue culture, sequencing library preparations (16S, whole genome, RNA sequencing), next generation sequencing (Miseq v2, v3 kits), yeast genetics, flow cytometry
\item \textbf{Bioinformatics} Differential expression analysis (DeSeq2), Metagenomics and meta-barcoding (Kraken2, Bracken, DADA2), Genome alignment (samtools, Bowtie2 etc), Flux Balance Analysis (cobra)
\item \textbf{Computing and graphics} Docker, Git, Emacs, \LaTeX, GIMP, Inkscape, Snakemake
\item \textbf{Mathematics} ODE modeling, Nonlinear dynamics, Numerical optimization
\end{itemize}

\cvsection{Publications}
\begin{etaremune}
\item \textbf{Jalihal, A.P.}, Degennaro, C., Jhuang, H-Y, Commins, N., Hamrick, S., Springer, M., 2021, Passive plasma membrane transporters play a critical role in
perception of carbon availability in yeast, \textit{bioRxiv}
\item \textbf{Jalihal, A. P.}, Kraikivski, P. , Murali, T. M. \& Tyson, J. J., 2020, Modeling and Analysis of the Macronutrient Signaling Network in Budding Yeast, \textit{Molecular Biology of the Cell}
\item Pratapa, A., \textbf{Jalihal, A. P.}, Law, J. N., Bharadwaj, A., \& Murali, T. M. (January 2020). Benchmarking algorithms for gene regulatory network inference from single-cell transcriptomic data. \textit{Nature Methods}, 1-8.
\item Pratapa, A., \textbf{Jalihal, A. P.}, Ravi, S. S., \& Murali, T. M. (August 2018). Efficient Synthesis of Mutants Using Genetic Crosses. In \textit{Proceedings of the 2018 ACM International Conference on Bioinformatics, Computational Biology, and Health Informatics} (pp. 53-62). ACM.
\end{etaremune}
\vfill
\begin{center}
Last updated on \textbf{\today}  
\end{center}




% \cvsection{Presentations}
% \begin{itemize}
% \item \ul{ASCB}, ,Washington DC, \textit{A potential growth rate sensing mechanism in budding yeast} \textbf{December 2022}
% \item \ul{Work In Progress Seminar, Department of Biological Sciences}, Blacksburg, VA, \textit{Applications of Modeling to Biological Decision Making} \textbf{December 2019}
%   \item \ul{Biology and  Medicine through Mathematics}, Richmond, VA, \textit{Macronutrient Signaling in S. cerevisiae} \textbf{May 2018}  
%   \item \ul{Biological Sciences Research Day}, Blacksburg, VA, \textit{Macronutrient Signaling in S. cerevisiae} \textbf{February 2018}
%   \item \ul{Computational Tissue Engineering Seminar}, Blacksburg, VA, \textit{Macronutrient Signaling in S. cerevisiae} \textbf{October 2017}
%   \item \ul{International Conference on Systems Biology}, Blacksburg, VA, \textit{Macronutrient Signaling in S. cerevisiae} \textbf{August 2017}
% \end{itemize}

% \cvsection{Teaching}
% \textit{Virginia Polytechnic and State University}
% \begin{itemize}
%   \item Graduate Teaching Assistant Integrated Science Curriculum II (ISC 1106), \textbf{Spring 2018}
%   \item Graduate Teaching Assistant Integrated Science Curriculum I (ISC 1105), \textbf{Fall 2017}
% \end{itemize}

% \cvsection{Service}
% \begin{itemize}
% \item Organized the Computational Tissue Engineering IGEP Graduate Seminar \textbf{Spring 2017 -- Fall 2019}  
% \item Contributed to maintenance of the GBCB website \textbf{Summer 2019}
% \item Volunteer at Kid's Tech University, Virginia Tech \textbf{Spring 2017}    
% \item Treasurer for the SPIC-MACAY chapter at Virginia Tech (Kala)  \textbf{Summer 2018 -- Spring 2020}  
% \end{itemize}



%\cvsection{Honors \& Awards}
% \begin{itemize}
%     \item Received accolades at Atos for Best Performance in team.
%     \item Received Best Debut Award at Atos.
%     \item Won 2nd Consolation Prize for paper presented on Cognitive Radio Networks.
%     \item Awarded with Narotam Sekhsaria Foundation Scholarship
%     \end{itemize}

% \cvachievement{\faTrophy}{}{Received accolades at Atos for Best Performance in team.}
% \cvachievement{\faTrophy}{}{Received Best Debut Award at Atos. }
% %\divider
% \cvachievement{\faInstitution}{}{Won 2nd Consolation Prize for paper presented on Cognitive Radio Networks.}
% %\divider
% \cvachievement{\faGraduationCap}{}{Got Selected in "Exclusive Scholar Program" during undergrad.}
% %\divider
% \cvachievement{\faDollar}{}{Awarded with Narotam Sekhsaria Foundation Scholarship}
%\cvsection{Strengths}

%\cvtag{Hard-working (18/24)} 
%\cvtag{Persuasive}
%\cvtag{Motivator \& Leader}

%\divider\smallskip

%\cvtag{UX}
%\cvtag{Mobile Devices \& Applications}
%\cvtag{Product Management \& Marketing}


%\divider

%\cvevent{B.S.\ in Symbolic Systems}{Stanford University}{Sept 1993 -- June 1997}{}

% \cvsection{Projects}
% \cvproject{Masked Face Detection for ATM}
% \begin{itemize}
% \item Developed a head classifier to detect masked faces in ATM to potentially prevent the event of robbery.
% \item Different camera angle, position, image quality, illumination and type of occlusion were the major challenges. Improved the existing accuracy by 20\%.
% \end{itemize}
% \smallskip
% \cvproject{Person Tracking}
% \begin{itemize}
% \item Developed, modified and implemented robust object tracker by combining motion and appearance information to learn deep association metrics.
% \end{itemize}
% \smallskip
% \cvproject{One Shot Learning}
% \begin{itemize}
% \item One shot learning is the promising approach to learn good feature when little data is available. 
% \item Achieved 92\% accuracy on omniglot dataset using Siamese network with Bayesian optimization.
% \end{itemize}
% \smallskip
% \cvproject{Automatic Defect Inspection of solar farm using drones}
% \begin{itemize}
% \item Regular inspection of solar farm due to its wide size is strenuous. 
% \item Developed a model to classify and localize defect on thermal images captured by drones.
% \end{itemize}
% \smallskip
% \cvproject{Anomaly detection using Auto-Encoders}
% \begin{itemize}
% \item Developed a model to learn regular patterns from sensor data and detect unusual pattern.
% \end{itemize}
% \smallskip
% \cvproject{Early Warning Fault Detection and Identification}
% \begin{itemize}
% \item Developed an LSTM based model to forecast and detect outlier from sensor data. 
% \item Further, classified the given signal into one of the type of outlier.
% \end{itemize}
% \smallskip
% \cvproject{Sentiment Analysis}
% \begin{itemize}
%     \item Used bag-of-words, pre-trained Embedding and simple as well as bi-directional LSTM techniques for Sentiment Analysis.
% \end{itemize}
% \cvproject{}
%%% Local Variables:
%%% mode: latex
%%% TeX-master: "jalihal"
%%% End:
